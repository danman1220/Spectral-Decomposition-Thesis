\begin{abstract}
In this work, applications of the spectral theorem for self-adjoint operators
in quantum mechanics are examined. While classical physics provides differential
equations (in the form of Newton's laws) that describe deterministic evolution
of phase space variables, quantum mechanics evolves the more abstract
wavefunction, an element of the quantum Hilbert space.  To find measurement
probabilities for a phase space variable (an "observable"), the observable is
identified with a self-adjoint operator on the corresponding Hilbert space. The
spectral decomposition of the quantized operator gives information on the values
possible for such an observable.  Furthermore, the different parts of the
spectrum of a quantum observable will be shown to correspond to different types
of states, an idea which will be demonstrated with concrete examples. Finally,
different formulations of the spectral theorem are explored, including the
projection-valued measure and direct integral approach. These different
formulations will yield further insights into physical understandings of
quantum mechanical states.
\end{abstract}
