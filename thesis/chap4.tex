\section{The Spectral Theorem}
Normal operators are quite special. We've already seen that a normal operator
will only have an approximate point spectrum, but a far more important
result--the spectral theorem--will better characterize the spectrum of a normal
operator. The spectral theorem will also give a very useful way of understanding
and computing with normal operators by expressing them as multiplication
operators. In this section, different formulations of the spectral theorem will
be explored, with the goal of understanding more concretely the relationship
between a normal operator and its spectrum.

\subsection{Spectral Theorem: The Finite Case}
The computational technique of matrix diagonalization is a very powerful tool
for working with matrices, since a diagonal matrix is much easier to compute
with.  Multiplying diagonal matrices corresponds to entrywise multiplication,
which is much easier to do than full matrix product calculations. Furthermore,
the components of a diagonalized matrix correspond to the eigenvalues of the
matrix.  These eigenvalues give valuable information on what the matrix does:
The largest eigenvalue gives information on the norm of the matrix, and an
eigenvalue of zero indicates that the matrix is noninvertible. Unfortunately,
not every matrix is diagonalizable.

Normal matrices are somewhat unique: the additional condition of normality gives
a somewhat surprising characterization of their diagonalization. This
characterization is stated in the \em spectral theorem \em for normal matrices.
\begin{theorem}[The Spectral Theorem in Finite Dimensions]
    Every normal matrix $A\in\mathbb{C}^{n\times n}$ is unitarily equivalent to
    a diagonal matrix. That is,
    \[
        A = UDU*
        \]
    For a unitary matrix $U$ and a diagonal matrix $D$. Furthermore, the columns
    of $U$ are the eigenvectors of $A$, and the components of $D$ are the
    corresponding eigenvalues.
\end{theorem}
Proofs of this theorem are standard in any introductory linear algebra
text (\cite[Ch. 17.3]{Lang1993} or \cite[p. 91]{MacCluer2009}).

Recall that a unitary matrix $U$ is one that satisfies $U^*U = UU^* = I$.
Unitary matrices are precisely the matrices that preserve inner
products (\cite[p. 39-40]{MacCluer2009}).
\begin{theorem}
    Let $U:T\to V$ be a linear transformation between inner product spaces. The
    following are equivalent:
    \begin{itemize}
        \itemsep0em
        \item $U$ is unitary.
        \item $U$ is invertible, and $U^* = U^{-1}$
        \item $U$ preserves inner products. That is, $\langle Ux, Uy\rangle =
            \langle x,y \rangle$
    \end{itemize}
\end{theorem}
Thus, unitary matrices are isometric isomorphisms, which are the isomorphisms in
the category of inner product spaces. In finite dimensions, a unitary matrix
corresponds to an orthonormal change of basis, where the columns of the unitary
matrix are the new basis vectors.

In light of this understanding of unitary matrices, another way of stating the
spectral theorem is as follows: any normal matrix $A$ has a set of eigenvectors
that form an orthonormal basis for the space it operates on.

Since the matrix $A$ corresponds to multiplication by an eigenvalue on the
corresponding eigenspace, the spectral theorem can again be restated:
\begin{theorem}[The Spectral Theorem in Finite Dimensions: Projection Version]
    Every normal matrix $A\in\mathbb{C}^{n\times n}$ is expressible as
    \[
        A(x) = \sum_{i=1}^n \lambda_i \langle x,e_i\rangle e_i
        \]
    where $\lambda_i$ is an eigenvalue, and $e_i$ is the associated
    eigenvector. Furthermore, the collection $\{e_i\}$ forms an orthonormal set.
\end{theorem}

If $\lambda_i \langle x,e_i\rangle e_i$ is interpreted as a projection onto the
subspace spanned by $e_i$, the spectral theorem looks like
\[
    A = \sum_{i=1}^n \lambda_i P_i
    \]
Where $P_i$ is projection onto the subspace spanned by the eigenvectors
associated with $\lambda_i$. In particular, the projections are all orthogonal
to each other: $P_i\circ P_j = 0$ if $i\not = j$.

The spectral theorem for matrices can be interpreted in two ways. First, it says
that a normal matrix is isometrically isomorphic to a diagonal matrix. That is,
normal matrices correspond to multiplication on some isomorphic space to the
space they operate on. This makes some amount of sense; normality is a condition
based on commutativity, and multiplication is inherently commutative.

Second, a normal matrix can be expressed as a linear combination of projection
operators. This formulation respects the invariance of the eigenspaces of $A$.
In particular, if $A$ is applied to a vector already in an eigenspace, the
vector stays in that eigenspace. The projections that make up $A$ characterize
the invariant subspaces of $A$, and the spectral theorem says that $A$ on each
of these subspaces is just multiplication by a (possibly different) scalar.

\subsection{Spectral Theorem: Extending to the Infinite Case}

As is the theme of Hilbert spaces, extending familiar results to infinite
dimensions will cause problems. It's somewhat surprising, then, that the
spectral theorem even holds at all in infinite dimensions. One of the biggest
differences between finite and infinite dimensional Hilbert spaces is one we've
already seen. The spectrum of a finite operator is always a collection of
disjoint points, whereas the spectrum of an infinite operator may have
infinitely many points, and could even be continuous!

Another characteristic of finite dimension Hilbert spaces that does not carry
over to infinite dimensions is the fact that (as long as they are both over the
same field) two Hilbert spaces are isomorphic if and only if they have the same
dimension. This fact was exploited implicitly in the first statement of the
finite dimensional spectral theorem by assuming the space the diagonal matrix
operated on was the same as the space the original matrix operated on. In
infinite dimensions, this will not hold, and we must be more careful in stating
which Hilbert space a normal operator operates as multiplication on.

Following the pattern from the finite dimensional spectral theorem statements,
there are two equivalent ways of stating the spectral theorem in general. One
version creates a Hilbert space for a normal operator where the operator is
isomorphic to multiplication, and the other version will express a normal
operator as an integral of projections.

\subsubsection{Projection-valued Measures and the Projection-valued Integral}
The projection-valued measure approach to the spectral theorem utilizes a bit of
measure theory in building a certain type of integral. Measure theory is the
cornerstone of Lebesgue integration, and gives a rigorous way to construct
integrals on more arbitrary spaces.

A \em measure \em $\mu$ on a space $X$ is a function from subsets of $X$ to the
positive real numbers (and possibly infinity) that satisfies
\begin{itemize}
    \itemsep0em
    \item $\mu(\emptyset) = 0$
    \item $\mu(A\cup B) = \mu(A) + \mu(B)$ for disjoint sets $A$ and $B$
\end{itemize}

Intuitively, a measure assigns to subsets their "weight". Thus, the weight of
the empty set is zero, and the weight of two disjoint sets is the sum of their
individual weights.

Integration on a measure space is first defined for characteristic functions.
The \em Lebesgue integral \em on a measure space $X$ is defined as the integral
for which
\[
    \int_X \chi_S d\mu = \mu(S)
\]
Then, functions on $X$ are approximated by sums of characteristic functions, and
with appropriate limits taken, a full integral can be defined.


    \begin{example}
        $X = \mathbb{R}$, and $d\mu = dx$ leads to the standard measure on the
        real number line: $\mu([a,b]) = b-a$. Here, the integral is the standard
        integral
            \[
                \int_{\mathbb{R}}f(x)dx
            \]
    \end{example}


    \begin{example}
        $X = \mathbb{N}$, and $d\mu$ is the counting ($\delta$) measure $\mu(S)
        = |S|$ leads to the integral
        \[
            \int_{\mathbb{N}}f(n)d\mu = \sum_{i=1}^{\infty}f(n)
        \]
        which is just familiar summation of sequences.
    \end{example}
    Further study of measure theory can be found in most any graduate analysis
    texts, such as \cite{Lang1993}.


This idea of measures can be extended. In the spectral theorem, associated with
each point in the spectrum was a projection onto a subspace of the Hilbert
space; that is, there was a function from subsets of the spectrum to projection
operators on the Hilbert space. This observation leads to the definition of a
\em projection-valued measure \em (\cite[Ch. 6.2]{MacCluer2009}).
\begin{definition}
    A \em projection-valued measure \em $E$ on a space $X$ is a function from
    subsets of $X$ to the set of orthogonal projections on a Hilbert space
    $\textbf{H}$ such that
    \begin{itemize}
        \itemsep0em
        \item $E(\emptyset)=0$ and $E(X) = I$
        \item For disjoint subsets $A$ and $B$, $E(A)(\textbf{H})\perp
            E(B)(\textbf{H})$
        \item For a sequence $\{S_i\}_{i=1}^{\infty}$ of disjoint subsets,
            $\sum_{i=1}^nE(S_i)h \to E(\cup_i S_i)h$ for each $h\in\textbf{H}$.
    \end{itemize}
\end{definition}

Using this definition of a measure, a projection-valued integral can be
constructed. In a similar way to regular Lebesgue integration, the integral is
defined first on characteristic functions
\[
    \int_X\chi_S dE = E(S)
\]
and then is extended to more arbitrary functions via limits. With this
definition of a projection-valued integral, we are ready to explore the first
restatement of the spectral theorem.

\begin{theorem}[The Spectral Theorem: Projection-Valued Integral]
    Let $A$ be a normal operator on a Hilbert space $\textbf{H}$, and let
    $\sigma(A)$ denote the spectrum of $A$. Then, there exists a unique
    projection-valued measure on $\sigma(A)$ such that
    \[
        A = \int_{\sigma(A)}zdE(z)
        \]
\end{theorem}
For a proof of this, see \cite[Ch. 6]{MacCluer2009}.

Of course, the difficulty will come in constructing the projection-valued
measure, but one is always guaranteed to exist.

To better understand this
result, consider the case where $\textbf{H}$ is finite-dimensional.
    \begin{example}
        Let $M = 
            \begin{bmatrix}
                \lambda_1 & 0 & \cdots & 0 \\
                0 & \lambda_2 & \cdots & 0 \\
                \vdots & \vdots & \ddots & 0\\
                0 & 0 & 0 & \lambda_n 
            \end{bmatrix}$

            Then, the spectral measure $E(S)$ is the $\delta$-measure on
            $\sigma(M)$ with $E(\lambda_i) = P_{\lambda_i}$, and the spectral
            theorem states that
            \[
                M = \int_{\sigma(M) = \{\lambda_i\}} zdE(z)
                  = \sum_{i=1}^n \lambda_i P_{\lambda_i}
                  \]
            which is a restatement of the familiar spectral theorem.
    \end{example}
For discrete points in a point spectrum, the projection-valued measure just
assigns each eigenvalue a projection onto its eigenspace. In a similar way to
how the counting measure worked on $\mathbb{N}$, the counting measure on
$\sigma_P(M)$ converted the integral into a familiar sum. This correspondence is
what motivates the projection-valued measure approach to the spectral theorem.
The simplest way to take a sum over a continuous interval is by converting it to
an integral.

Let's consider a more difficult example.

    \begin{example}
        Let $M_x$ be the familiar multiplication operator on $L^2([0,1],dx)$.
        It is easy to show that $\sigma(M_x) = \sigma_{AP}(M_x) = [0,1]$. If
        $\lambda$ is not in $[0,1]$, then the operator of multiplication by
        $\frac{1}{x-\lambda}$ inverts $(M_x-\lambda I)$, and thus $\lambda$ is
        not in $\sigma(M_x)$.

        The spectral measure $E(S)$ for an interval $S$ is given as $M_{\chi_S}$
        for $\chi_S$ the characteristic function on $S$. That is, a projection
        in this space is multiplication by a characteristic function.

        See \cite[Ex. 6.11]{MacCluer2009} for more details on the construction of this
        spectral measure.
    \end{example}
\subsubsection{The Direct Integral}

Recall the alternative definition of the spectral theorem, which stated that
every normal matrix is isometrically isomorphic to a multiplication operator.
Here, the infinite-dimensional pathologies will come into play in defining the
space on which a normal operator is isometrically isomorphic to multiplication.

To begin with, we define an additional structure on a collection of Hilbert
spaces. For a measure space $(X,\mu)$, let
$\{\textbf{H}_{\lambda}\}_{\lambda\in\textbf{H}}$ be a collection of Hilbert
spaces, one for each point $\lambda$ in $X$. Define a \em section \em of the
collection to be a function from $X$ to the collection of Hilbert spaces such
that $S(\lambda) \in \textbf{H}_{\lambda}$. In essence, $s$ picks out one vector
from each $\textbf{H}_{\lambda}$. Furthermore, we can give the set of all such
sections an inner product structure by defining
\[
    \langle s_1,s_2 \rangle = \int_X \langle s_1(\lambda),s_2(\lambda)\rangle
    d\mu(\lambda)
\]

Finally, we define the \em direct integral \em of a collection of Hilbert spaces
$\{\textbf{H}_{\lambda}\}$ to be the inner product space of (equivalence classes
of) sections on $\{\textbf{H}_{\lambda}\}$, denoted
\[
    \int_X^{\oplus}H_\lambda d\mu(\lambda)
\]
It is easily shown (\cite[Ch. 7.3]{Hall2013})
that this new space is complete, and thus is a Hilbert space.

\begin{example}
    Let $X$ be a measure space, and let $H_{\lambda} = \mathbb{C} \forall
    \lambda\in X$. Then,
    \[
        \int_X^{\oplus}H_\lambda d\mu(\lambda) = L^2(X)
    \]
\end{example}

\begin{example}
    Let $X = \{a_1, a_2, \hdots, a_n\}$ with the counting measure. Then
    \[
        \int_X^{\oplus}H_{a_i} d\mu(a_i) = \oplus_{i=1}^n H_{a_i}
        \]
    That is, the counting measure again reduces integration to a familiar sum
\end{example}

The spectral theorem can now be stated.

\begin{theorem}[The Spectral Theorem: Direct Integral]
    Given a normal operator $A$ on a Hilbert space, there exists a measure $\mu$
    on the spectrum $\sigma(A)$ of $A$, and Hilbert spaces $H_{\lambda}$ for
    $\lambda\in\sigma(A)$ such that $A$ is unitarily equivalent to
    multiplication on the direct integral
    \[
        \int_{\sigma(M)}^{\oplus}H_\lambda d\mu(\lambda)
    \]
    That is, $(UAU^*)s(\lambda) = \lambda s(\lambda)$ for $s$ in the direct
    integral.
\end{theorem}

The spaces $H_{\lambda}$ in general will not be subspaces of the original
Hilbert space, but instead will represent more generalized eigenspaces of the
operator $A$. However, in the case of finite dimensions, each $H_{\lambda}$ is a
subspace, and the original spectral theorem is recovered.

\begin{example}
    Let $M$ be a normal matrix with distinct eigenvalues. Then, let $\mu$ be the
    counting measure on $\sigma(A)$, and let $H_{\lambda}$ be the
    one-dimensional eigenspace associated with the eigenvalue $\lambda$. Then,
    the direct integral becomes
    \[
        \int_{\sigma(M)}^{\oplus}H_\lambda d\mu(\lambda) = \oplus_{i=1}^n
        H_{\lambda}
        \]
    and the spectral theorem states that $M$ is unitarily equivalent to
    multiplication on each subspace. In other words, $M$ is unitarily equivalent
    to a diagonal matrix.
\end{example}
