\section{Conclusion}

Quantum mechanics relies heavily on functional analysis and Hilbert space
mathematics, and without a good understanding of the intricacies of
infinite-dimensional Hilbert spaces, apparent pathologies can easily confuse and
mislead.

The spectrum of a quantum observable corresponds to the possible measurement
results of that observable, so understanding the spectrum of observables is key
in understanding a quantum system. The naive approach to understanding the
spectrum is to solve the eigenvalue equation $\hat{f}x=\lambda x$ in a way
analogous to finite-dimensional linear algebra. While this works in certain
cases--like the Hamiltonian for the infinite square well--that have only point
spectra, most observables will not be that simple. Trying to solve the
eigenvalue equation in the finite square well setup yielded near-nonsensical
results of "eigenstates" that were no longer in the Hilbert space of the system!

These pathologies can be resolved by exploring the spectral theorem for
infinite-dimensional operators more closely. Two formulations of the spectral
theorem provide key insight into the behavior of the spectrum of a normal (or
self-adjoint) operator. Specifically, the projection-valued measure approach and
the direct integral approach offer different but equivalent ways of interpreting
the spectrum.

The projection-valued measure approach to the spectral theorem constructs a
"differential projection", or a projection-valued measure, that takes into
account the structure of the spectrum in constructing the projection operators
onto the eigenspaces for a normal operator. For points in the spectrum that are
isolated, this differential projection becomes an actual projection operator,
but for intervals of points in the spectrum, the differential projection has to
be integrated over to yield an actual projection. This reflects the apparent
pathology encountered in the naive solution to the finite square well energies:
for energies greater than zero, a single energy is not enough to generate a
projection. However, when taking an interval of energies, an invariant subspace
can be constructed that remains in the Hilbert space.

The direct integral approach constructs "generalized eigenspaces" for each point
in the spectrum, and assigns to the whole spectrum a measure that takes the
structure of the spectrum into account. Here, the isolated points have a point
mass measure, and their associated eigenspace is a true invariant subspace of
the original Hilbert space. Again, for the continuous part of the spectrum, no
single "generalized eigenspace" has positive measure, but an interval of them
integrate to an actual invariant subspace.

Both formulations say the same thing: the point spectrum of an operator has true
eigenvalues and eigenstates, whereas the approximate point spectrum of the
operator will not have actual eigenstates for any particular eigenvalue. Rather,
some positive-measure set of approximate spectrum points must be considered for
the theory to work.
