\chapter{Introduction to Quantum Mechanics}

The early 1900's brought about a distinct revolution in our understanding of
physics. Certain issues in small-scale thermal physics were becoming more
apparent as the experiments and the theory became more precise.
The total energy emitted by a black body radiator was
predicted to be infinity, while observations guaranteed that the average iron
rod was not, in fact, a source of infinite energy. Furthermore, the
photoelectric effect, a phenomenon in which a metal subjected to certain
frequencies of light would emit electrons, was not well-understood. It was
thought at the time that these small problems were the last remaining questions
to be answered in physics, but they would prove to be signs of a much greater
misunderstanding of the physical world.

Planck's solution to the black-body problem was intriguing. He achieved
finite predictions to black body energy radiation by assuming that the
energy of the radiation came in discrete quantized packets of energy, $E=h\nu$,
where $h$ is Planck's constant. While Planck used this method merely as a
calculation tool, Einstein would later show that such an assumption has a more
physical interpretation.

The problem with the photoelectric effect was as follows: the electrons that
eject from the metal have energy values that only depend on the \em{frequency},
not the \em{intensity} of the light. Furthermore, below a certain color
frequency, no electrons would be emitted at all, regardless of intensity. If one
were to view the incoming light waves as pure waves, the energy delivered to the
electrons would be directly proportional to the power of the electromagnetic
wave. However, the average power of such a sinusoid is given as 
\[
    P_{avg} = \frac{Af}{2}
\]
Where $A$ is the amplitude, and $f$ is the frequency of the light. Under such a
treatment of light (assuming electrons need a minimum amount of power to be
ejected), any frequency of light--given a high enough amplitude--can eject
photons. However, experiments showed that this was not the case.

Einstein's solution to the problem was to assume that the energy delivered to
the metal was not continuous--it came in discrete quantized packets he called
photons. In this interpretation, the intensity of light describes how many of
these photons are emitted per second, and the color or frequency of the light
describes the amount of energy contained in a single photon. The dynamics of the
photoelectric effect are determined by the interactions of the individual
photons with the metal. This explains why the frequency of light, not the
intensity, dictated the energy of the electrons emitted.

Both Planck's black body solution and Einstein's photoelectric solution hint at
a more complicated physical model than the classical theory described.
This quantization of certain physical properties (like the amount of
energy delivered by light) represents one of the fundamental differences between
classical and quantum mechanics.

\section{Measurement in Quantum Mechanics}
%TODO-measurement in classical mechanics vs quantum
%In classical mechanics, a particle's dynamics are described by Newton's laws.
%Through the equation
%\[
%    \Del V = \frac{dp}{dt}
%\]
%A particle's phase space variables (position and velocity) are completely
%determined by initial conditions and the potential. Furthermore, there is no
%mathematical limitation on the values of the initial conditions; given a
%potential and a desired state, one can choose initial conditions that
%will evolve to that state.

Quantum mechanics, on the other hand, does not hold the phase variables in the
same regard. The fundamental evolution object is the \em{wavefunction}, a unit
vector in a Hilbert space, usually $L^2$ on an appropriate domain. The phase
variables of position and momentum then take on the role of (potentially
unbounded) operators on this Hilbert space. In this view, the wavefunction
evolves deterministically following the Schrodinger equation:
\[
    -i\frac{\del \psi}{\del t} = \hat{H}\psi
\]

Where $\hat{H}$ is the quantum mechanical Hamiltonian operator.


\section{An Example: The Infinite Square Well}

The most pertinent difference between classical physics and quantum physics lies
in the quantization of the state variables of the system. To illustrate this
difference, consider the simple setup of a particle of mass $m$ in a
one-dimensional infinite square well potential
\[
    V(x) = 
    \begin{cases}
        0 & \text{if } x\in[-a,a]\  \infty & \text{else}
    \end{cases}
\]

%TODO-ISW example

\section{The Finite Square Well: Continuous vs Discrete Spectra}

Consider a similar potential function, the finite square well
\[
    V(x) = 
    \begin{cases}
        0 & \text{if } x\in[-a,a]\  V_0 & \text{else}
    \end{cases}
\]
for some positive potential height $V_0$

