\section{Spectral Analysis of Linear Operators}
In finite dimensional linear algebra, diagonalization is a powerful tool used to
understand linear operators. One of the key results in linear algebra is the
spectral theorem, which gives a method of diagonalizing a matrix in terms of its
eigenvalues and eigenvectors. Since the study of quantum mechanics focuses on
finding eigenvalues of the quantized linear operators, it is natural to ask
whether or not the spectral theorem still holds in the infinite-dimensional
case. While infinite dimensionality introduces some difficulties, there are a
few equivalent statements of the spectral theorem that work in infinite
dimensions, and each will provide further insight into the nature of the
"eigenvalues" of linear operators.

\subsection{The Spectrum of an Operator}
In finite dimensions, the \em spectrum \em of a matrix is the set of eigenvalues
associated to that matrix. That is, for a matrix $A \in \mathbb{R}^{n\times n}$,
the spectrum of $A$ is given as
\[
    \sigma(A) = \{\lambda \in \mathbb{C} : Ax=\lambda x \text{ for some } x \in
    \mathbb{R}^{n}\}
\]

Another way of restating this is that $\sigma(A)$ is the set of all (complex)
scalars $\lambda$ such that the matrix $A-\lambda I$ has nontrivial kernel.

Of course, in finite dimensions, having a nontrivial kernel is equivalent to
being noninvertible, so the spectrum of a matrix $A$ is best described as the
set of all $\lambda$ such that $A-\lambda I$ is not invertible. By defining the
spectrum in terms of invertibility, it is easy to extend the concept to
arbitrary Hilbert spaces.

For a general operator $T \in \mathscr{B}(H)$ (where $\mathscr{B}(H)$ is the set
of bounded linear operators on a Hilbert space \textbf{H}), it is best to define
the \em spectrum \em of the operator as
\[
    \sigma (T) = \{\lambda \in \mathbb{C} : T-\lambda I \text{ is not
    invertible}\}\]

It is easy to see that this definition coincides with the usual definition in
finite dimensions, but in an infinite dimensional Hilbert space, things might be
more complicated.

The problem lies in the fact that there are more ways to fail invertibility in
infinite dimensions than just having nontrivial kernel. There are three general
ways to fail invertibility in a Hilbert space, and each failure will lead to a
different subclassification of the spectrum. The subclassifications of the
spectrum will be called the spectral partitions of $\sigma(T)$.

The spectral partitions are as follows (\cite[p. 115]{MacCluer2009}):

\begin{enumerate}
    \itemsep0em
    \item $T-\lambda I$ has nontrivial kernel (the \em point spectrum\em).
    \item $T-\lambda I$ is not bounded below (the \em approximate point
        spectrum\em).
    \item $T-\lambda I$ does not have dense range (the \em compression
        spectrum\em).
\end{enumerate}

Let's consider each of these cases individually.

\subsubsection{The Point Spectrum}
The \em point spectrum \em is all $\lambda$ such that $T-\lambda I$ has a
nontrivial kernel. More succinctly, $\lambda \in \sigma_P(T)$ if and only if
there exists some vector $x\in \textbf{H}$ such that $(T-\lambda I)x = 0$, or
$Tx=\lambda x$.  This is the familiar eigenvalue equation seen in an
introductory linear algebra course, and leads to the conclusion that for any
linear operator on a finite dimensional Hilbert space (like $\mathbb{C}^n$), the
spectrum of the operator is entirely a point spectrum.
\begin{example}
    Consider the matrix
    \[
        M = 
            \begin{bmatrix}
                \lambda_1 & 0 & \cdots & 0 \\
                0 & \lambda_2 & \cdots & 0 \\
                \vdots & \vdots & \ddots & 0\\
                0 & 0 & 0 & \lambda_n 
            \end{bmatrix}
    \]
    This has eigenvalues $\lambda_n$, and thus its spectrum is
    \[
        \sigma(M) = \sigma_P(M) = \{\lambda_n\}
    \]
\end{example}

\subsubsection{The Approximate Point Spectrum}
The point spectrum is actually a subset of the larger \em approximate point
spectrum \em, which consists of all $\lambda$ such that the operator $T-\lambda
I$ is not bounded below. That is, $\lambda \in \sigma_{AP}(T)$ if and only if
there exists some sequence of unit vectors $\{h_n: h_n \in \textbf{H}\}$ for
which
\[
    ||(T-\lambda I)h_n|| \to 0.
\]
The vectors $h_n$ are commonly referred to as "approximate eigenvectors".

It is easy to see that if $\lambda\in\sigma_P(T)$, then
$\lambda\in\sigma_{AP}(T)$, since one can take for $h_n$ the constant sequence
of the eigenvector corresponding to $\lambda$. Then, $||(T-\lambda I)h_n|| = 0$
and $\lambda\in\sigma_{AP}(T)$.

Because of this, it is convenient to define the \em strict approximate point
spectrum \em $\sigma_{SAP}$ to be $\sigma_{AP}\setminus\sigma_P$.

\subsubsection{The Compression Spectrum}
If $T-\lambda I$ fails to have dense range, $\lambda$ is said to be part of the
\em compression spectrum \em of $T$. This part of the spectrum will not be
explored further: the following theorem will motivate why.
\begin{theorem}
    For a normal operator $T$, the spectrum of $T$ is entirely approximate point
    spectrum.
\end{theorem}
\begin{proof}
    Let $\lambda$ be in $\sigma(T)$ for a normal operator $T$, and consider the
    ways in which $T-\lambda I$ can fail to be invertible.

    If the kernel of $T-\lambda I$ is nontrivial, then $\lambda\in\sigma_{P}(T)$,
    and thus $\lambda\in\sigma_{AP}(T)$.

    Consider, then, the case where the kernel of $T-\lambda I = \{0\}$. Now,
    since $||Ax|| = ||A^*x||$ for any normal operator (see \cite[prop.
    2.14]{MacCluer2009}),
    $||(T-\lambda I)x|| = ||(T^* - \overline{\lambda}I)x|| \not= 0$, so the
    kernel of $T^* - \overline{\lambda}I$ is trivial as well.
    By the Fredholm alternative (\cite[p. 350]{Olver2014}),
    \[
        \overline{\text{Ran}(T-\lambda I)} = \text{Ker}(T^* -
        \overline{\lambda}I)^{\perp} = \{0\}^{\perp} = \textbf{H}
        \]
    And thus the range of $T-\lambda I$ is dense, and $\lambda$ is not in the
    compression spectrum of $T$.
\end{proof}

All that being said, there are easy examples of operators with a compression
spectrum.
\begin{example}
    Consider the Hilbert space $l^2(\mathbb{R)}$, the space of
    square-summable sequences. Sequences in this space are commonly represented
    as "infinite tuples" of the form $(a_1, a_2,\hdots)$. The right shift
    operator $U$, defined as
    \[
        U((a_1,a_2,\hdots)) = (0,a_1,a_2,\hdots)
        \]
    is a bounded linear operator on $l^2(\mathbb{R})$. Furthermore,
    the operator $U-0I = U$ does not have dense range: it misses the entire
    dimension spanned by $(1,0,0,\hdots)$. Thus, $0\in\sigma_C(U)$.
\end{example}

For more examples of operators and their spectra, see
\cite[Ch. 6.5]{Kubrusly2011}.
