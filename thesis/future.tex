\section{Future Work}
One of the clearest issues in this analysis is the fact that most quantum
observables will be unbounded. While the spectral theorem still holds in the
unbounded case, greater care must be taken to avoid the pathologies that
unbounded operators bring with them. An introduction to unbounded operators can
be found in \cite[Ch. 9]{Hall2013}.

While this work focused on the point spectrum/approximate point spectrum
decomposition of the spectrum of an operator, there are other ways to partition
the spectrum. \cite{Kubrusly2008} examines a finer partition of the spectrum,
and it has not yet been explored whether these finer partitions result in
phenomenological differences in quantum mechanics.

Another aspect of quantum mechanics that was not touched on in this work is the
quantization of more arbitrary phase variables. While \cite[Ch. 13]{Hall2013}
examines different quantization schemes, the effect of these quantizations on
the spectrum of their resultant observables should be examined further.

It was also hinted at in the direct integral approach that the measure assigned
to the spectrum contained information regarding the multiplicity or degeneracy
of points in the spectrum. Degeneracy is an important topic in quantum
mechanics (see \cite[Ch. 6.2]{griffiths2005}), and the spectrum measure from the
direct integral may provide further insight into characterizing degeneracies.
