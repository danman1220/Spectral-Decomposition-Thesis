\section{Quantum Revisited}
With the new tools of the spectral theorem, we can now better understand the
more subtle intricacies of quantum mechanics. Recall that one of the rules of
quantum mechanics gives a way of computing a probability distribution for an
observable from its spectrum. Specifically, a consequence of rule 3 is that the
possible values an observable can take coincide with the spectrum of the
observable. Furthermore, the relative probabilities of each value are given by
the length of the projections of a state onto the "eigenspace" of that value.
However, it was clear in the finite square well example that some parts of the
spectrum may not have actual eigenspaces. The spectral theorem gives a way to
better understand what the general projections and invariant spaces look like
for these more complicated operators.

\subsection{The Finite Square Well Revisited}
Let's reconsider the finite square well potential in quantum mechanics. Recall
that the relevant Hilbert space for this problem is $L^2(\mathbb{R})$, and the
Hamiltonian operator is of the form
\[
    \hat{H}(x) =
    \begin{cases}
        \frac{-\hbar^2}{2m}\frac{d^2}{dx^2} - V_0& \text{if } x\in[-a,a]\\
        \frac{-\hbar^2}{2m}\frac{d^2}{dx^2} + &\text{else}
    \end{cases}
\]

The Hamiltonian can be shown to have a spectrum that looks like
\[
    \sigma(\hat{H}) = \{E_1, E_2,\hdots,E_n\} \cup (0,\infty)
\]
Explicit, square-integrable functions could be found that lie in the kernel of
$\hat(H) - EI$ for $E < 0$, which are of the form
\[
    \psi(x) =
    \begin{cases}
        C_1e^{\sqrt{\epsilon}x}& \text{if } x\in(-\infty,-a]\\
        C_2\cos(\sqrt{v-\epsilon})& \text{if } x\in[-a,a]\\
        C_3e^{-\sqrt{\epsilon}x}& \text{if } x\in [a, \infty)
    \end{cases}
\]
Where $\epsilon = -\frac{2mE}{\hbar^2}$ and $v=\frac{2mV_0}{\hbar^2}$, and
$C_1$, $C_2$, and $C_3$ were constants set to normalize $\psi$. Since such
eigenfunctions exist, the points $E_n$ where $E_N<0$ are part of the point
spectrum $\sigma_P(\hat{H})$.

Showing that $(0,\infty)$ is in the spectrum of $\hat{H}$ requires a bit more
work. In our first attempt at finding the eigenfunctions for $E>0$, we saw that
the only solutions for the eigenvalue equation $\hat{H}\psi = E\psi$ were
complex exponentials of the form $e^{\pm ikx}$ for
$k=\frac{\sqrt{2mE}}{\hbar}$. Since these solutions are not in
$L^2(\mathbb{R})$, they do not lie in the kernel of $\hat{H}-EI$, and are not
sufficient to show that $E$ is a part of the spectrum of $\hat{H}$. However, the
point spectrum is not the only spectral partition available to us. It may be the
case that $E$ is in the approximate point spectrum of $\hat{H}$ without being in
the point spectrum. As it turns out, this is the case.

\begin{theorem}
    For all $E>0$, $E\in\sigma_{AP}(\hat{H})$, where $\hat{H}$ is the
    Hamiltonian for the finite square well acting on the Hilbert space
    $L^2(\mathbb{R})$.
\end{theorem}

\begin{proof}
    Fix $E>0$, and let $f_n$ be a $C^{\infty}$ function on $\mathbb{R}$ such
    that
    \[
        f_n(x) =
        \begin{cases}
            1 & x\in(-n,n)\\
            0 & x\in(-(n+1),n+1)
        \end{cases}
        \]
    Then, let $\psi_n(x) = \frac{e^{ikx}f_n(x)}{||e^{ikx}f_n(x)||}$ for
    $k=\frac{\sqrt{2mE}}{\hbar}$. This defines a sequence of unit length states
    for which 
    \[
        ||(\hat{H}-EI)\psi_n(x)||\to 0
        \]
    Thus, the operator $\hat{H}-EI$ is not bounded below, and
    $E\in\sigma_{AP}(\hat{H})$
\end{proof}

We find that the spectrum of $\hat{H}$ is made up of finitely many disjoint
points $E_n$ below zero (the point spectrum of $\hat{H}$, and the entire
positive real number line (the approximate point spectrum of $\hat{H}$).

Constructing the projection-valued measure for this operator is fairly
straightforward, using the results of the previous analysis. For simplicity,
$E$ will represent an energy (an element of the spectrum of $\hat{H}$), and $F$
will be the projection-valued measure.

For the point spectrum, 
\[
    dF(E) = P_E
\]
where $P_E$ is the orthogonal
projection onto the one dimensional subspace of the state with energy $E$.

For the approximate point spectrum, one can interpret $dF(E)$ to be a
projection onto the two dimensional subspace spanned by the "states"
\[
    \begin{aligned}
        \psi_{E}(x) &= e^{-ikx} \\
        \psi_{E}(x) &= e^{ikx}
    \end{aligned}
\]

Thus, the spectral theorem states that
\[
    \begin{aligned}
        \hat{H} &= \int_{\sigma{\hat{H}}}zdF(z)\\
                &= \sum_{n} E_n P_{E_n} + \int_0^{\infty}EdF(E)
    \end{aligned}
\]

Now, the problem with interpreting the complex exponentials as eigenstates is
much clearer. Each $E_n$ yielded a true eigenstate, which allowed for their
projection-valued measure to be the "point mass" measure. That is, the discrete
points had weight on their own. However, in the continuum $(0,\infty)$, the
measure is more akin to the Lebesgue measure on the reals. Single points have no
weight, but instead it takes a whole interval of points to integrate over to
have positive mass. Here, a single $E>0$ does not yield a projection, but an
interval of energies will yield a projection onto a subspace in $L^2$.

Now, let's construct the direct integral for this operator. Similar to the
projection-valued measure approach, each part of the spectrum will be taken
separately.

The measure on the point spectrum is the counting measure, so that part of
the integral becomes
\[
    \int_{\sigma_P(\hat{H})}^{\oplus}\textbf{H}_{E_n}d\mu(E) =
    \oplus_{i=1}^{n} \textbf{H}_{E_n}
\]
Where $\textbf{H}_{E_n}$ is the one dimensional subspace of the state with
energy $E_n$.

The measure on the approximate point spectrum will be similar to the Lebesgue
measure, and the integrand $\textbf{H}_E$ can be shown to be the two-dimensional
subspace of complex exponentials $e^{ikx}$ and $e^{-ikx}$ for
$k=\frac{\sqrt{2mE}}{\hbar}$.

Thus, the Hilbert space for which $\hat{H}$ acts as multiplication is
\[
    \oplus_{i=1}^n \textbf{H}_{E_n} \oplus
    \int_{\sigma_{AP}(\hat{H})}^{\oplus} H_Ed\mu(E)
\]

Again, we see that the problem with treating the complex exponentials as
eigenstates has to do with their measure. In this formulation, the distinction
is even more clear, as $\mu(E)=0$ for a single $E>0$. Thus, while the complex
exponentials do form a sort of "invariant subspace" in the sense that they are
part of the direct integral, their measure being zero hints at the fact that
they are not part of the original Hilbert space. Rather, a positive-measure
interval of these invariant subspaces is needed to properly fit them in $L^2$.

Note that in each case, the point spectrum of the operator behaved well and led
to the familiar theory of eigenvalue decomposition, while the approximate point
spectrum led to pathologies that had to be resolved. The point spectrum was a
set of isolated points, and the approximate point spectrum ended up being a
continuum.

This observation yields a more general result:
\begin{theorem}
    Isolated points in the spectrum of a normal operator are part of the point
    spectrum.
\end{theorem}

That is to say, isolated energies of a quantum system always yield definite
eigenstates. Furthermore, the contrapositive asserts that the approximate point
spectrum will not have any isolated points.
