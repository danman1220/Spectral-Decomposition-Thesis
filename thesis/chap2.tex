\section{Hilbert Space Basics}
In quantum mechanics, each system has an associated Hilbert space where the
states live, and each observable has an associated (potentially unbounded)
operator on that Hilbert space. It is necessary, then, to understand what
Hilbert spaces are and how to work with them.

\subsection{Hilbert Space Definitions}
A \em complex Hilbert space \em \textbf{H} is a vector space over $\mathbb{C}$
along with an inner product such that \textbf{H} is complete with respect to the
induced metric
\[
d(x,y) = \langle x-y,x-y\rangle
\]
\cite[p. 8]{MacCluer2009}

Here, \textbf{H} being complete means that every Cauchy sequence in \textbf{H}
converges in \textbf{H}. This condition is important, as it allows for limits to
be taken without worry of accidentally leaving the space.

The requirement that the vector space be over $\mathbb{C}$ is not necessary for
a general Hilbert space, but most Hilbert spaces encountered in quantum
mechanics will be over $\mathbb{C}$.

Most familiar vector spaces are, in fact, Hilbert spaces. The familiar vector
space $\mathbb{C}^n$ is a vector space over $\mathbb{C}$ that is complete with
respect to its inner product, since $\mathbb{C}$ itself is complete, and the
metric on $\mathbb{C}^n$ respects the standard metric on $\mathbb{C}$. However,
a Hilbert space need not be finite dimensional.

\begin{example}[$L^2$ as a Hilbert Space]
An important class of Hilbert spaces is the $L^2$ spaces of square integrable
functions. Given a positive %positive?
measure space $(X,\mu)$, let $L^2(X,\mu)$ be the collection of all equivalence
classes of measurable functions $f: X \to \mathbb{C}$ that satisfy
\[
    \int_X |f|^2d\mu < \infty
\]
where $f\sim g$ if and only if $\int_X |f-g|^2d\mu = 0$. As a consequence of
this, changing $f$ on a set of measure zero will not change its equivalence
class.

To define an inner product on $L^2(X,\mu)$, we set
\[
    \langle f,g \rangle = \int_X f\overline{g}d\mu
\]
For a more detailed construction of $L^2$ spaces, see \cite[p.
    181-185]{Lang1993}.
\end{example}


\subsubsection{Orthonormal Bases}
A natural way to express elements of a Hilbert space is in reference to a basis.
Preferably, this basis should also reflect the inner product structure of the
Hilbert space, and the most useful way to do so is by using an orthonormal
basis.

An orthonormal set is a set $\mathscr{E}$ of vectors in a Hilbert space
\textbf{H} where
\begin{itemize}
    \item $\forall e \in \mathscr{E}, |e| = 1$
    \item $\forall e,f \in \mathscr{E}, \langle e,f\rangle = \delta_{ef}$
\end{itemize}
where $\delta_{ef}$ is the Kronecker Delta on $e$ and $f$.

An orthonormal basis is defined as a maximal orthonormal set. A quick
application of Zorn's lemma (\cite[p. 19, 50]{MacCluer2009}) shows that every
Hilbert space has an orthonormal basis.

For an arbitrary vector $\psi$ in \textbf{H} with ONB $\{e_n\}$, its ONB
expansion in terms of $\{e_n\}$ is
\[
    \psi = \sum_{n=0}^{\infty} \langle \psi,e_n \rangle e_n
\]

\subsection{Hilbert Space Duality: Functionals and Representation}
Every vector space $V$ has a corresponding dual space of linear functionals on
$V$, denoted $V^*$. This is the space of all linear functions
$\phi:V\to\mathbb{R}$. In the case of a vector space with norms, it is
additionally required that the function be bounded with respect to the norm
\[
    ||\phi|| = \sup_{x\in V}\{|\phi(x)| : |x| \leq 1\}
\]

When studying Hilbert spaces, it is important to understand the structure of its
dual space. To that end, the Riesz representation theorem will give an easy way
to understand the dual space of a Hilbert space in terms of the Hilbert space
itself.

\begin{theorem}[Riesz Representation]
    Any linear functional on a Hilbert space \textbf{H} is given by an inner
    product with a unique fixed vector $h_0 \in \textbf{H}: \phi(x) =
    \langle h_0, x \rangle$, and the norm of the functional is $||h_0||$.
\end{theorem}
\cite[p. 17]{MacCluer2009}

In other words, there is an isometric isomorphism between $H$ and $H^*$. Because of this,
Hilbert spaces are said to be \em self-dual\em.

\subsection{Operators on a Hilbert Space}
For any pair $X$ and $Y$ of Hilbert spaces, there is another normed vector space
that is of great importance: the space of bounded linear operators from $X$ to
$Y$. For an operator $T:X\to Y$ to be linear, it must satisfy
\[
    T(\alpha x_1 + \beta x_2) = \alpha T(x_1) + \beta T(x_2)
\]
for vectors $x_1, x_2$ and scalars $\alpha, \beta$. Furthermore, for $T$ to be
bounded, there must exist some positive constant $C$ such that
\[
    ||T(x)||_Y \leq C||x||_X
\]

Bounded linear operators are a natural generalization of the matrix operators
seen in a linear algebra course, but also include many other useful operators,
such as integral operators, multiplication operators, and other such linear
objects. (See \cite[p. 32-34]{MacCluer2009} for more examples).

\begin{example}
    Consider a matrix $M\in \mathbb{R}^{n\times m}$ as a linear transformation
    from $\mathbb{R}^n$ to $\mathbb{R}^m$ defined as $M(x) = Mx$. This is a
    bounded linear operator from $\mathbb{R}^n$ to $\mathbb{R}^m$. Linearity is
    obvious from the definition of matrix multiplication, since $M(x+y) = Mx +
    My$. Furthermore, the matrix is also bounded, since $\forall x \in X$ we
    have
    \[
        ||Mx|| \leq ||M||||x||
    \]
    where the vector norm is the standard Euclidean norm, and the matrix norm is
    the induced norm, which is guaranteed to be finite. %TODO-citation?
\end{example}

\begin{example}
    Consider the operator
    $T:L^2([0,1])\to L^2([0,1])$ given by $Tf(x) = \int_0^x f(y)dy$.
    This operator is
    easily seen to be both bounded and linear.
    More generally, given a measurable function $k:X\times X \to \mathbb{C}$ on
    a measure space $(X,\mu)$, the \em integral operator\em
    \[
        \int_X k(x,y)f(y)d\mu(y)
    \]
    from $L^2(X)$ to itself is a bounded linear operator. Thus, the familiar
    integral operator from differential equations is just a special case of a
    bounded linear operator on some Hilbert space.
\end{example}


\begin{example}
    The derivative operator $\frac{d}{dx}$ on $L^2([0,1])$ is a linear operator,
    since
    \[
        \frac{d}{dx}(\alpha f + \beta g) =
        \alpha\frac{df}{dx} + \beta\frac{dg}{dx}.
    \]

    However, this operator is not bounded. In fact, it's only defined on a
    (dense) subset of $L^2([0,1])$, the space of differentiable functions with a
    square-integrable derivative. For a detailed treatment of this operator, see
    \cite[p. 127-128, Ch. 9]{Hall2013}.
\end{example}

\subsection{Adjoints of Operators}
Given a Hilbert space, there is a natural symmetry present in the inner product.
For a bounded operator $A$ from a Hilbert space \textbf{H} to itself, we define
the \em adjoint of $A$ \em to be the unique bounded operator $A^*$ such that,
$\forall x,y \in \textbf{H}$,
\[
    \langle Ax,y\rangle = \langle x, A^*y\rangle
\]
Such a unique bounded operator is guaranteed to exist as a consequence of Riesz
Representation (see \cite[Thm. 2.12]{MacCluer2009}).
In the case of $A=A^*$, the operator is said to be \em self-adjoint\em, and in
the case of $AA^*=A^*A$, the operator is said to be \em normal\em. The condition
of normality is a powerful property, and will be the condition necessary for an
operator to satisfy the spectral theorem. Furthermore, we have already seen that
every quantum observable is self-adjoint, and thus is normal. The rest of this
paper will focus on normal operators.
